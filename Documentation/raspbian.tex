\section{Uruchomienie w środowisku \textsl{Raspbian}}
Algorytm działania całego systemu jest następujący:
\begin{enumerate}
  \item Sprawdzenie zawartości konfiguracji \texttt{WiFi},
  \item jeżeli istnieje wpis o określonym \texttt{SSID}	
    następuje połączenie z nim,
  \item jeżeli nie został przydzielony address \texttt{IP} (sieć nie jest aktywna)
    następuje uruchomienie własnego \texttt{AP}
    \footnote{od ang. Access Point -- punkt dostępu 
    do sieci (w danym przypadku \texttt{WiFi})} oraz
    aplikacji konfiguracyjnej,
  \item jeżeli adres został przydzielony -- uruchomienie \texttt{Telegram Bot'u},
    aplikacji pomiarowej oraz aplikacji \texttt{WEB} wyświetlającej dane.
\end{enumerate}

Powyższy algorytm został napisany w języku \texttt{Bash}.
\begin{lstlisting}[basicstyle=\ttfamily\small, language=bash, frame=single,
  caption={Uruchomienie procesów w systemie \textsl{Raspbian}}]
#!/bin/sh
working_dir="/home/pi/ZWS_rasp_tg/"
sudo service wpa_supplicant restart
sudo wpa_supplicant -D wext -i wlan0 \
  -c /etc/wpa_supplicant/wpa_supplicant.conf -B
WPA_PID=$!
sleep 2
sudo dhclient -1
if [ $? -eq 0 ]
then
	echo "Connected"
	cd $working_dir && python3 ./Telegram_Bot/rasp_bot.py &
	cd $working_dir && python3 ./Web_app/app.py &
else
	echo "No internet"
	sudo kill $WPA_PID
	sudo service wpa_supplicant stop
	sudo iptables-restore /etc/iptables.ipv4.nat
	sudo ifup wlan0
	sudo service hostapd start
	sudo service dnsmasq start
	cd $working_dir && python ./network_auth/app.py
fi
\end{lstlisting}
W celu uruchomienia danego skryptu podczas startu sytemu bez
konieczności ręcznego wywołania, został napisany \texttt{service}:
\begin{lstlisting}[basicstyle=\ttfamily\small, language=bash, frame=single,
  caption={\texttt{service} wywołujący skrypt}]
#!/bin/sh
#
### BEGIN INIT INFO
# Provides:          checkInet
# Required-Start:    $remote_fs $syslog
# Required-Stop:     $remote_fs $syslog
# Should-Start: 
# Should-Stop:
# Default-Start:     2 3 4 5
# Default-Stop:      
# Short-Description: Starting script as 'pi' user 
# Description:
### END INIT INFO
su pi /home/pi/ZWS_rasp_tg/start.sh
\end{lstlisting}
