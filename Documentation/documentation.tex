\documentclass[a4paper]{article}
\usepackage{polski}
\usepackage{geometry}[top=2cm, left = 2.5 cm, right = 2.5cm]
\title{
	\textbf{Zastosowanie systemów wbudowanych}\\
	\textit{Raspberry Pi - Telegram Bot i web-aplikacja dla śledzenia klimatu otoczenia}
	}
\date{}

\begin{document}
\begin{table}
	\vspace{-2cm}
	\begin{tabular}{rl}
		Autorzy & \\
			\hline
		Kostiukov Oleksii 	& 231972\\
		Uladzimir Lipski	& 238961\\
		Vikatr Hasiul 		& 231862\\[0.4cm]
			\hline
		Prowadzący 	& Dr inż. Marek Woda \\
		Termin		& Środa, godzina: 13:15\\
	\end{tabular}
\end{table}

{\let\newpage\relax\maketitle}
\maketitle
\thispagestyle{empty} %no page number for this page

\maketitle
\newpage
\section{Cel projektu}
	Celem danego projektu jest wykorzystanie platformy \textit{Raspberry} 
	do realizacji \textit{Telegram bot'u} oraz punktu pomiarowego.

	\paragraph{Telegam bot} jest aplikacją wykorzystującą interfejs aplikacji \textit{Telegram}
	w celu komunikacji z wybranymi użytkownikami (którzy posiadają możliwość komunikacji ze stworzonym
	\textit{bot'em}).

	\paragraph{Punkt pomiarowy.} Platforma \textit{Raspberry} umożliwia podłączenie licznych
	czujników, dane z których można gromadzić na urządzeniu lub wysyłać do serwerów zdalnych.
	W danym projekcie zostaną podłączone czujniki:
	\begin{enumerate}
		\item temperatury,
		\item wilgotności,
		\item światła
	\end{enumerate}
	dane z których będą przechowywane na urządzeniu w celu przetwarzania i 
	wyświetlania na stronie \textbf{WEB} w postaci interaktywnego wykresu.
	
	Dodatkowo dany zbiór danych zostanie wykorzystany przez \textit{Telegram bot}
	w celu powiadomienia użytkownika o aktualnych danych.
\end{document}
