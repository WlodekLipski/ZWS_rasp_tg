\section{Łączenie z siecią \texttt{WiFi}}
W celu połączenia z siecią użytkownika \texttt{WiFi} o standardzie 
\texttt{WPA} lub \texttt{WPA2} została napisana aplikacja,
która pobiera dane sieci:
\begin{itemize}
  \item SSID\footnote{od ang. Service set identifier - identyfikator sieci},
  \item hasło.
\end{itemize}
które zostają przekazane do aplikacji:
\begin{itemize}
  \item \texttt{wpa\_passphrase} -- generowanie klucza dla połączenia,
  \item \texttt{wpa\_supplicant} -- w celu połączenia z siecią za pomocą
    klucza i \texttt{SSID}
\end{itemize}
\begin{lstlisting}[frame=single, basicstyle=\ttfamily\small, language=python,
caption={Pobieranie danych z form za pomocą metody \texttt{POST}}]
@app.route('/', methods=['GET', 'POST'])
def index():
    if request.method == 'POST':
        subprocess.Popen(['./switch_network.sh %s %s'
			 % (request.form['ssid'],
			 request.form['password'])], shell=True)
        return render_template('index.html')
    else:
        return render_template('index.html')
\end{lstlisting}
Generowanie klucza dla połączenia zostało napisane w języku \texttt{Bash}:
\begin{lstlisting}[frame=single, basicstyle=\ttfamily\small, language=bash,
caption={Generowanie klucza połączenia}]
#!/bin/bash
#user input for SSID
#creating a config file for network
echo "ctrl_interface=DIR=/var/run/wpa_supplicant GROUP=netdev"\
  | sudo tee /etc/wpa_supplicant/wpa_supplicant.conf
echo "update_config=1" \
  | sudo tee -a /etc/wpa_supplicant/wpa_supplicant.conf
wpa_passphrase $1 $2 \
  | sudo tee -a /etc/wpa_supplicant/wpa_supplicant.conf
sudo reboot now
\end{lstlisting}
Po restarcie system będzie próbował połączyć się ze wskazaną siecią.

