\section{Generowanie i przechowywanie danych}
Generowanie danych jest oparte o informacje uzyskiwaną z 
czujników pomiarowych:
\begin{itemize}
  \item \texttt{DHT11} - czujnik temperatury i wilgotności,
  \item \texttt{BHT1750} - czujnik natężenia światła.
\end{itemize}
Zakresy pomiarowe czujniku \texttt{DHT11} stanowią pewne ograniczenie
na jego zastosowanie:
\begin{itemize}
  \item Temeperatura -- [0,50] $\pm 2^\circ$ o rozdzielczości $1^\circ$,
  \item wilgotność -- [20,90]\% \texttt{RH} - wilgotność względna. 
    Jest to stosunek rzeczywistej wilgoci w powietrzu do maksymalnej jej ilości, 
    którą może utrzymać powietrze w danej temperaturze.
    Rozdzielczość -- $1\%$.
\end{itemize}
Z powodu powyższych ograniczeń, dany czujnik można stosować w warunkach, które
nie wymagają większej precyzji, na przykład -- pomieszczenia biurowe.

Czujnik natężenia światła posiada szeroki zakres pomiarowy -- $[1,65535] lx$,
z rozdzielczością 1 lub 4 $lx$ w zależności od wybranego trybu pracy. 
W danej implementacji została wybrana rozdzielczość w 1 $lx$.

\subsection{Przechowywanie danych}
\subsection{Podłączenie czujników}
