\section{Telegram Bot}
\textsl{Telegram Bot} jest aplikacją bazującą się na 
\texttt{API}\footnote{od ang. Application Programming Interface -- programistyczny interfejs aplikacji} 
udostępnionego przez \texttt{Telegram} w celu tworzenia \texttt{Bot'a} -- aplikacji odpowiadającej
na określone żądania użytkownika w określony sposób. Jest to pewien rodzaj usługi, którą możemy 
stworzyć i wdrożyć w danym messengerze.

Uługą danego projektu jest udostępnianie wyników pomiarowych, które są zbierane
przez niezależną aplikację, działającą na platformie \texttt{Raspberry Pi 3B+}.

Proces tworzenia \texttt{Bot'a} jest opisany na stronie -- \texttt{https://core.telegram.org/bots}

W danej implementacji \texttt{Bot'a} zostały zrealizowane następujące funkcjonalności:
\begin{itemize}

\item Wgląd do wyników pomiarowych -- odczyt ostatnich danych pomimarowych:
Użytkownik posiada możliwość zdefiniowania ilości ostatnich wyników pomiarowych.
\begin{lstlisting}[frame=single, basicstyle=\ttfamily\small, language=python,
caption={Komedna do pobierania określonej ilości ostatnich wyników pomiarowych}]
/tail <size>
\end{lstlisting}
gdzie \texttt{<size>} określa ilość pomiarów.
\begin{lstlisting}[frame=single, basicstyle=\ttfamily\small, language=python,
caption={Odpowiedź na wprowadzoną komendę -- \texttt{/tail}}]
def rasp_tail(update, context):
    tail_size = 10
    if len(context.args) != 0:
        tail_size = int(context.args[0])

    if tail_size < 0:
        tail_size *= -1
    data = read_csv('data.csv').tail(tail_size)
    for name in TRUE_ARGS.values():
        msg = name
	      +':\n'
	      +re.sub(' {2,}',' ',
		      data.loc[:,name].to_string(index=False))
        update.message.reply_text(msg)

\end{lstlisting}
\newpage

\item Przeglądanie wyników pomiarowych w postaci graficznej -- wykresy typu \textsl{histogram}:
Użytkownik posiada możliwość wyboru typu danych pomiarowych.
\begin{lstlisting}[frame=single, basicstyle=\ttfamily\small, language=python,
caption={Komedna do pobierania wyników określonego typu, w postaci graficznej}]
/pic <type>
\end{lstlisting}
    gdzie \texttt{<type>} określa jeden z możliwych typów pomiarowych: 

    \texttt{temp, humid, light} -- \texttt{temperatura, wilgotność, natężenie światła}.

\begin{lstlisting}[frame=single, basicstyle=\ttfamily\small, language=python,
caption={Odpowiedź na wprowadzoną komendę -- \texttt{/pic}}]
def rasp_pic(update, context):
    picture_size = None
    picture_type = None
    if len(context.args) > 1:
        picture_type = context.args[0]
        picture_size = int(context.args[1])
        file_name = create_plot(TRUE_ARGS[picture_type], picture_size)
        context.bot.send_photo(chat_id=update.effective_chat.id,
			       photo=open(file_name, 'rb'))
    elif len(context.args) == 1:
        picture_type = context.args[0]
        file_name = create_plot(TRUE_ARGS[picture_type])
        context.bot.send_photo(chat_id=update.effective_chat.id,
			       photo=open(file_name, 'rb'))
    else:
        update.message.reply_text('Invalid args')
\end{lstlisting}
\end{itemize}
